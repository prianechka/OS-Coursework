\section{Технологический раздел}

\subsection{Выбор языка и среды программирования}
В качестве языка программирования был выбран язык C. 
На этом языке реализованы все модули ядра и драйверы операционной системы Linux.

В качестве среды разработки была выбрана Visual Studio Code. 

\subsection{Реализация изменения яркости}
На листинге 5 представлена реализация изменения яркости.

Функция \textit{brightness\_update} записывает новое значение яркости в файл, а \textit{backlight\_change} рассчитывает в соответствии с наступившем событием это значение.

\FloatBarrier
\begin{lstinputlisting}[language=C, caption=Реализация изменения яркости, linerange = {37-55},
	basicstyle=\footnotesize\ttfamily, frame=single, breaklines=true]{../src/brightness.c}
\end{lstinputlisting}
\FloatBarrier

\subsection{Реализация обработчика прерываний от мыши}
В приложении А представлена реализация обработчика прерываний от мыши.

Для определения нажатой клавиши используется поле \textit{data} из структуры \textit{mouse\_t}. 
Второй байт поля отвечает за тип нажатой клавиши, поля с третьего по шестой включительно используются для вычисления новых координат.

\subsection{Реализация драйвера}
В листинге 6 представлена реализация функции \textit{probe}.

\FloatBarrier
\begin{lstinputlisting}[language=C, caption=Реализация функции probe, linerange = {126-206},
	basicstyle=\footnotesize\ttfamily, frame=single, breaklines=true]{../src/controller.c}
\end{lstinputlisting}
\FloatBarrier

\clearpage

На листинге 7 представлена реализация функции \textit{disconnect}.
\FloatBarrier
\begin{lstinputlisting}[language=C, caption=Реализация функции disconnect, linerange = {208-221},
	basicstyle=\footnotesize\ttfamily, frame=single, breaklines=true]{../src/controller.c}
\end{lstinputlisting}
\FloatBarrier

\clearpage

На листинге 8 представлена реализация функции \textit{init} для драйвера.
\FloatBarrier
\begin{lstinputlisting}[language=C, caption=Реализация функции init, linerange = {260-295},
	basicstyle=\footnotesize\ttfamily, frame=single, breaklines=true]{../src/controller.c}
\end{lstinputlisting}
\FloatBarrier

\clearpage

На листинге 9 представлена реализация функции \textit{exit} для драйвера.
\FloatBarrier
\begin{lstinputlisting}[language=C, caption=Реализация функции exit, linerange = {297-304},
	basicstyle=\footnotesize\ttfamily, frame=single, breaklines=true]{../src/controller.c}
\end{lstinputlisting}
\FloatBarrier

\subsection{Makefile}
На листинге 10 представлена реализация Makefile.
\FloatBarrier
\begin{lstinputlisting}[language=C, caption=Makefile, linerange = {},
	basicstyle=\footnotesize\ttfamily, frame=single, breaklines=true]{../src/Makefile}
\end{lstinputlisting}
\FloatBarrier